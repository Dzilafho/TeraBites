\documentclass[english]{article}


\usepackage{graphicx}
\usepackage{grffile}
\usepackage[T1]{fontenc}
\usepackage{babel}
\usepackage{float}
\usepackage{ragged2e}



\title{\scshape\Large Team members}
\author{
	NG Maluleke\\
	\texttt{13229908}
	\and
	D Mulugisi\\
	\texttt{13071603}
	\and
	C Nel\\
	\texttt{14029368}
	\and
	LE Tom\\
	\texttt{13325095}
}


\graphicspath{{Pictures/}}

\begin{document}

	
	\begin{figure}
		\includegraphics[width=\linewidth]{up_logo.png}
	\end{figure}
	
	\begin{center}
	 \line(1,0){370}
	\\[0.2cm]
    {\scshape\Large Functional specification \par}
	\vspace{0.1cm}
	\line(1,0){370}
	\\[0.8cm]
	
	{\scshape\large Project name: Arcane Arcade\par}	
	\vspace{1cm}
	{\scshape\large Client: Tony vd Linden\par}
	\vspace{1cm}
	{\scshape\large Team name: Terabites\par}
	\vspace{1cm}
	{\let\newpage\relax\maketitle}
	\end{center}
	
	
	\pagenumbering{gobble}
	\newpage
	\tableofcontents

	\pagenumbering{arabic}
	\newpage
	
	\section{Introduction}
		 The name \textit{Arcane Arcade} references the esoteric language that users will have to use to interact with the system. It also highlights the gamification approach that makes it a fun activity. The platform will be used to test prospective employees in a fun gamified way. It will indicate in which programming area the user is likely to be better suited, by looking at how the user completed the challenges, such as whether the user used any hints and the badges that the user earned.

	\section{Vision}
	The vision is to provide a fun and easily accessible platform that can be used to gauge the programming aptitude and capabilities of existing or future software developers using a custom esoteric programming language.
	
	\newpage
	\section{Functional Requirements and Application Design}
	
		\subsection{Use Case Prioritization}
		This section outlines the three-level prioritization of the system's required functionality.
		\subsubsection{Critical}
		\begin{itemize}
	  		\item \textbf{User Authentication} - This is critical as it limits the system's access only to people who have been authorized. Only registered users are then able to 		complete the challenges and/or modify the challenges together with authorizing more users. No one can then access any part of the system unless they have been registered and authorized to request such services.
	  		
			\item \textbf{Player Interaction} -  Interaction is critical as the whole system's core functionality revolves around it. The players have to be able to view and take challenges that are open to them at that time.
			
			\item \textbf{Code Compilation} - Code compilation is a critical use case as other instances of Player Interaction revolve around it. Where a Player interacts by typing code the system needs to be able to compile that code.
	   \end{itemize} 
		
		\subsubsection{Important}
		\begin{itemize}
	  		\item \textbf{Challenge Management} - This is considered important because this is where users who have management privileges will configure the structure of the challenges such as the number of levels and questions. The managers will also be able to set hints and other items related to the challenges.
	  		\item \textbf{Level Management} - This use case is important and it consists of the functionality of adding and removing levels on challenges and this can only be done by the administrator. 
	  		\item \textbf{Question Management} - This use case is important and it consists of the functionality of adding and removing questions.
	  		\item \textbf{User Management} - This use case is important as it deals with adding and removing users to the system. 
	   \end{itemize} 
			
		\subsubsection{Nice-To-Have}
		\begin{itemize}
	  		\item \textbf{Achievements} - This is considered nice to have as it for promoting the game and to get more people to play.
	  		\item \textbf{Social Media} - This is considered nice to have. It is to showcase in-game achievements on social media. It strengthens competition and markets the game.
	   \end{itemize}
	   
		\newpage
		\subsection{System Modules}
		The system has built as a series of separate modules. This approach makes it possible to add more modules at a later stage and to modify one module without interfering with the others. It then allows the system to be pluggable and integrable.\newline \newline
		
		
		
		\includegraphics[width=14cm,height=14cm,keepaspectratio]{ArcaneArcade.jpg}
		
		\newpage
		\subsubsection{User Authentication Module}
		This module provides the functionality which validate the user credential to check if they match up with the access that they request. The most crucial part and main focus of the authentication module is to verify if a user has access to certain pages given the type of user they are. This then improves the system's security and keeps the data confidential as it controls who has access to which functionality. \newline \newline
		
		\includegraphics[width=14cm,height=14cm,keepaspectratio]{Authenticate.jpg}
		
		\begin{itemize}

		\item \textbf{validateDetails} - Any user who uses the system has to login first. The user provides a username and a password which is compared with the credentials that are stored in the system database. If the credentials match then the user is given access to the system.
		
		\item \textbf{loadProfile} - After a user's credentials have been verified, their profile has to be loaded. This involves gathering data about the game which includes their progress, badges, hints, and credits to buy hints.
		\end{itemize}

		\newpage
		\subsubsection{User Management Module}
		This module focuses on the ability to grant and revoke user access to the system. Only the super user has access to the module and it can be used to add, remove and update user accounts. \newline
		
		\includegraphics[width=14cm,height=14cm,keepaspectratio]{userManagement.jpg}
		\begin{itemize}

	  		\item \textbf{addUser} - The manager has the ability to add more user accounts via the user management page thereby allowing them to access the system. Users are sent an email containing their login credentials after they are added to the system. The service contract and activity diagram for addUser are shown below.
	  		\newpage
	  		

				\begin{center}
	  			Service Contract
	  		    \end{center}
	  		    
	  			\begin{figure}[H]
	  				\begin{center}
	  					\includegraphics[scale=0.25]{AddUserContract.jpg}
	  				\end{center}
	  				\caption{Add new user to the system}
	  				
	  			\end{figure}
	  		
	  			
	  			\begin{center}
	  				Activity diagram
	  			\end{center}
	  			
	  			\begin{figure}[H]
	  				\begin{center}
	  					\includegraphics[scale=0.3]{addUser1.jpg}
	  				\end{center}
	  				\caption{The activity diagram for addUser}
	  				
	  			\end{figure}
	  
	  		\newpage
	  		\item \textbf{updateUser} - The updateUser use case falls under User as depicted in the above diagram. The manager has the ability to modify user accounts items via the user management page. The service contract and activity diagram for updateUser are shown below.
	  		
	  		
	  			\begin{center}
	  				Service Contract
	  			\end{center}
	  			
	  			\begin{figure}[H]
	  				\begin{center}
	  					\includegraphics[scale=0.25]{UpdateUserContract.jpg}
	  				\end{center}
	  				\caption{Update User on the system}
	  				
	  			\end{figure}
	  		
	  		
	  		\newpage	
			\item \textbf{removeUser} - The removeUser use case falls under User as depicted in the above diagram. The manager has the ability to remove any user account via the user management page thereby restricting them from accessing the system. The service contract and activity diagram for removeUser are shown below.
			
			\begin{center}
				Service Contract
			\end{center}
			
			\begin{figure}[H]
				\begin{center}
					\includegraphics[scale=0.25]{removeUserlContract.jpg}
				\end{center}
				\caption{Remove user from the system}
				
			\end{figure}
			
			
			\begin{center}
				Activity diagram
			\end{center}
			
			\begin{figure}[H]
				\begin{center}
					\includegraphics[scale=0.3]{RemoveUser1.jpg}
				\end{center}
				\caption{The activity diagram for removeUser}
				
			\end{figure}
		\end{itemize}		
		
		\subsubsection{Challenge Management Module}
		Only super users(management accounts) have access to this module.It provides the ability to add, remove and modify the structure of challenges. It allows the manager to modify the challenge at any of its three levels which are challenge, level and question. The modifications that the manager can do are add, remove and update any of the three levels of a challenge. \newline
		
		\includegraphics[width=15cm,height=15cm,keepaspectratio]{challengeManagement.jpg}
		
		\begin{itemize}
			\item \textbf{addChallenge} -  The addChallenge use case falls under Challenge Management module as depicted in the above diagram.  The addChallenge submodule allows the administrator to add challenges to the system via the Challenge Management page. The service contract and activity diagram for addChallenge are shown below. \newpage
			\begin{center}
				Service Contract
			\end{center}
			
			\begin{figure}[H]
				\begin{center}
					\includegraphics[scale=0.25]{AddChallengeContract.jpg}
				\end{center}
				\caption{Add challenge to the system}
				
			\end{figure}
			
			
		 \item \textbf{removeChallenge} -  The removeChallenge use case falls under Challenge Management module as depicted in the above diagram.  The removeChallenge submodule allows the administrator to remove challenges from the system via the Challenge Management page. The service contract and activity diagram for removeChallenge are shown below.
			\newpage
		 	\begin{center}
		 		Service Contract
		 	\end{center}
		 	
		 	\begin{figure}[H]
		 		\begin{center}
		 			\includegraphics[scale=0.25]{removeChallengeContract.jpg}
		 		\end{center}
		 		\caption{Remove challenge from the system}
		 		
		 	\end{figure}
		 	
		 	
		 	 \item \textbf{viewChallenges} -  The viewChallenges use case falls under Challenge Management module as depicted in the above diagram. The viewChallenges submodule allows the administrator to view challenges that are on the system via the Challenge Management page. The service contract and activity diagram for viewChallenges are shown below.
		 	
		 	 \begin{center}
		 	 	Service Contract
		 	 \end{center}
		 	 
		 	 \begin{figure}[H]
		 	 	\begin{center}
		 	 		\includegraphics[scale=0.2]{viewChallengesContract.jpg}
		 	 	\end{center}
		 	 	\caption{View all the challenges on the system}
		 	 	
		 	 \end{figure}
		 	 \newpage
		 	 	\begin{center}
		 	 		Activity diagram
		 	 	\end{center}
		 	 	
		 	 	\begin{figure}[H]
		 	 		\begin{center}
		 	 			\includegraphics[scale=0.2]{ViewChallenges.jpg}
		 	 		\end{center}
		 	 		\caption{The activity diagram for ViewChallenges}
		 	 		
		 	 	\end{figure}
		 	
			
		
		\end{itemize}
		\subsubsection{Level Management Module}
		Only super users(management accounts) have access to this module.It provides the ability to add, remove and view levels of a particular challenge.
		
				\includegraphics[width=14cm,height=14cm,keepaspectratio]{levelManagement.jpg}
		
	    \begin{itemize}
	    	
	    \item \textbf{addLevel} -  The addLevel use case falls under Level Management module as depicted in the above diagram.  The addLevel submodule allows the administrator to add levels to challenges on the system via the level Management page. The service contract and activity diagram for addLevel are shown below.	

	    \begin{center}
	    	Service Contract
	    \end{center}
	    
	    \begin{figure}[H]
	    	\begin{center}
	    		\includegraphics[scale=0.2]{AddLevelContract.jpg}
	    	\end{center}
	    	\caption{Add level to a challenge}
	    	
	    \end{figure}
	    
	    \item \textbf{removeLevel} -  The removeLevel use case falls under Level Management module as depicted in the above diagram.  The removeLevel submodule allows the administrator to remove levels from challenges on the system via the level Management page. The service contract and activity diagram for addLevel are shown below.	
	    
	     \begin{center}
	     	Service Contract
	     \end{center}
	     
	     \begin{figure}[H]
	     	\begin{center}
	     		\includegraphics[scale=0.2]{removeLevelContract.jpg}
	     	\end{center}
	     	\caption{Remove level from a challenge}
	     	
	     \end{figure}
	    
	
 
 	    \item \textbf{viewLevels} -  The viewLevels use case falls under Level Management module as depicted in the above diagram.  The viewLevels submodule allows the administrator to view all the levels from challenge on the system via the level Management page. The service contract and activity diagram for viewLevels are shown below.	
 	    
 	     \begin{center}
 	     	Service Contract
 	     \end{center}
 	     
 	     \begin{figure}[H]
 	     	\begin{center}
 	     		\includegraphics[scale=0.4]{ViewLevelsContract.jpg}
 	     		s
 	     	\end{center}
 	     	\caption{View all levels on a challenge}
 	     	
 	     \end{figure}
 	     
 	    \end{itemize}
	
		\subsubsection{Question Management Module}
		The Question Management Module can only be used by Managers.It provides the ability to add, remove and modify the structure of Questions. The manager can modify any question, add a new one or delete an existing one. It has the following Services.
		
		\includegraphics[width=10cm,height=10cm,keepaspectratio]{questionManagement.jpg}
		
		
			\begin{itemize}
		  		\item \textbf{addQuestion} -  The manager has the ability to add questions to an existing Level via the management page. The service contract and activity diagram for addQuestion are shown below.
		  		
		  		\begin{center}
		  			Service Contract
		  		\end{center}
		  		
		  		\begin{figure}[H]
		  			\begin{center}
		  				
		  				\includegraphics[width=10cm,height=10cm,keepaspectratio]{addQuestionContract.jpg}
		  			\end{center}
		  			\caption{View all levels on a challenge}
		  			
		  		\end{figure}
		  		
		  
	  		
	  		\item \textbf{removeQuestions} -  The manager has the ability to remove Questions though the management portal. This questions belong to levels which in turn belong to challenges. The service contract and activity diagram for removeQuestion are shown below.
	  		
	  			\begin{center}
	  				Service Contract
	  			\end{center}
	  			
	  			\begin{figure}[H]
	  				\begin{center}
	  				
	  					\includegraphics[width=10cm,height=10cm,keepaspectratio]{removeQuestion.jpg}
	  				\end{center}
	  				\caption{View all levels on a challenge}
	  				
	  			\end{figure}
	  		
			\item \textbf{viewQuestions} - The viewQuestions use case provides the manager the ability to get a list of all Questions from all Levels on the system. The service contract and activity diagram for viewQuestions are as follows.
			
				\begin{center}
					Service Contract
				\end{center}
				
				\begin{figure}[H]
					\begin{center}
						\includegraphics[width=10cm,height=10cm,keepaspectratio]{viewQuestionsContract.jpg}
					\end{center}
					\caption{View all levels on a challenge}
					
				\end{figure}
			
	
			\end{itemize}
		

		\subsubsection{Compiler Module}
		This module is the backbone of the system. It is responsible for everything that has to do with the user's code. It compiles, executes and displays output that has been generated by the user code.
		
		\subsubsection{Player Interaction Module}
		This module houses the system's main functionality. The main reason of having the system is to allow the users to interact with the system and complete the offered challenges. This use case diagram indicates the functionality around interacting with the system as a player.
		
		
		\section{Domain Model}
		 \includegraphics[width=15cm,height=15cm,keepaspectratio]{domainModel v2.png}
	\iffalse	
	
		\subsubsection{User interaction}
		\begin{description}
			\item[$\bullet$] The user may log in to the system if registered.   
			\item[$\bullet$] User may view available challenges.
			\item[$\bullet$] User may choose a challenge to do.
			\item[$\bullet$] User may receive a hint for the active challenge, if the user can afford it.
			\item[$\bullet$] After doing a challenge, the user can then send his/her code to be compiled.
			\item[$\bullet$] The user can finish a challenge if his/her code compiled and returned correctly.
		\end{description}
		
		\subsubsection{Admin interaction}
		\begin{description}
			\item[$\bullet$] The administrator may add or remove users from the system.
			\item[$\bullet$] The admin may change esolang keywords.
			\item[$\bullet$] The admin may add, remove or change challenges.
		\end{description}

		
		\subsection{Use case}
		Use Case are described below.
		
		
		
		\subsubsection{The Login}
		\begin{itemize}
	
		
		\item Pre-Conditions
			\begin{enumerate}
				\item The user is an admninistrator
				\item The user is a player
				\item The login credentials are valid
			\end{enumerate}
		\item Post-Conditions
			\begin{enumerate}
			\item The user is successfully logged into the system
						
			\end{enumerate}
		
			
		

\end{itemize}

		\subsubsection* {The login use case diagram}
		\begin{itemize}
			\item Description\\
			This use case will be used by the REST clients, specifically mobile app(tablet app), to login into the system.
		\end{itemize}
		\includegraphics[width=14cm,height=6cm,keepaspectratio]{login.jpg}\\
		
		
		\subsubsection{The Add User}
		
			\begin{itemize}
	
		
		\item Pre-Conditions
			\begin{enumerate}
				\item User must be an administrator
				\item No user with the same email exists.
			\end{enumerate}
		\item Post-Conditions
			\begin{enumerate}
			\item User is added to the system
			\item User( Player) is emailed a link with the username
						
			\end{enumerate}
	

\end{itemize}
		
		
		\subsubsection* {The add user use case diagram}
		\begin{itemize}
			\item Description\\
			This use case will be used by the REST clients, specifically the web client since the maintenence portal will be web based, to add a user.
		\end{itemize}
		\includegraphics[width=14cm,height=6cm,keepaspectratio]{addUser.jpg}
		
		
		\subsubsection{The Remove User}
		
			\begin{itemize}
	
		
		\item Pre-Conditions
			\begin{enumerate}
				\item User must be an administrator
				\item User(player) must exist.
			\end{enumerate}
		\item Post-Conditions
			\begin{enumerate}
			\item User is removed from the system.
						
			\end{enumerate}
	

		\end{itemize}
		
		
		\subsubsection* {The remove user use case diagram}
		\begin{itemize}
			\item Description\\
			This use case will be used by the REST clients, specifically the web client since the maintenence 					portal will be web based, to remove a user.
		\end{itemize}
		
	
		\includegraphics[width=14cm,height=6cm,keepaspectratio]{removeUser.jpg}
		
				
		
		\subsubsection{The Change Keywords}
		
			\begin{itemize}
	
		
		\item Pre-Conditions
			\begin{enumerate}
				\item User must be logged in as an administrator
			\end{enumerate}
		\item Post-Conditions
			\begin{enumerate}
			\item Keywords changed
						
			\end{enumerate}
		

		\end{itemize}
		
		
		
		
		
		\subsubsection* {The change keywords use case diagram}
		\begin{itemize}
			\item Description\\
			This use case will be used by the REST clients, specifically the web client since the maintenence portal will be web based, to change keywords on the system.
		\end{itemize}
		\includegraphics[width=14cm,height=6cm,keepaspectratio]{keyWords.jpg}	
		
		
			\subsubsection{The manage challenges usecase.}
			
			\begin{itemize}
				
				
				\item Pre-Conditions
				\begin{enumerate}
					\item User must be logged in as an administrator
				\end{enumerate}
				\item Post-Conditions
				\begin{enumerate}
					\item Challenge added
					\item Challenge changed
					\item Remove User
					
				\end{enumerate}
				
				
			\end{itemize}	
			
			\subsubsection* {The manage challenges usecase.}
			\begin{itemize}
			   \item Description\\
				This use case will be used by the REST clients, specifically the web client since the maintenence portal will be web based,to add challenges,change       challenges and to remove challenges.
			\end{itemize}
			\includegraphics[width=14cm,height=6cm,keepaspectratio]{Challenge.jpg}	
		
		\subsubsection{The View Challenges}
		
			\begin{itemize}
	
		
		\item Pre-Conditions
			\begin{enumerate}
				\item User must be logged in as player
				
			\end{enumerate}
		\item Post-Conditions
			\begin{enumerate}
			\item All the available challenges are displayed on the system
						
			\end{enumerate}
	

		\end{itemize}
		
		
		\subsubsection* {The view challenges use case diagram}
		\begin{itemize}
			\item Description\\
			This use case will be used by the REST clients, specifically the mobile app(tablet app), to view all available challenges on the system.
		\end{itemize}
		
	
		\includegraphics[width=14cm,height=6cm,keepaspectratio]{viewChallenge.jpg}
		
		\subsubsection{The Start Challenge}
		
			\begin{itemize}
	
		
		\item Pre-Conditions
			\begin{enumerate}
				
				\item User must be logged in as a player
			\end{enumerate}
		\item Post-Conditions 
			\begin{enumerate}
			\item The challenge is started and timed
						
			\end{enumerate}
	

		\end{itemize}
		
		
		\subsubsection* {The start challenge use case diagram}
		\begin{itemize}
			\item Description\\
			This use case will be used by the REST clients, specifically the mobile app(tablet app), to start a challenge on the system.
		\end{itemize}
		
	
		\includegraphics[width=14cm,height=6cm,keepaspectratio]{startChallenge.jpg}
		
		\subsubsection{The Buy Hints}
		
			\begin{itemize}
	
		
		\item Pre-Conditions
			\begin{enumerate}
				
				\item User must be logged in as a player
				\item Player should have started the challenge
			\end{enumerate}
		\item Post-Conditions
			\begin{enumerate}
			\item A hint is bought and displayed to assist the player, while player coins are decreased.
						
			\end{enumerate}
	

		\end{itemize}
		
		
		\subsubsection* {The buy hints use case diagram}
		\begin{itemize}
			\item Description\\
			This use case will be used by the REST clients, specifically the mobile app(tablet app), to buy hints 
			during gameplay.
		\end{itemize}
		
	
		\includegraphics[width=14cm,height=6cm,keepaspectratio]{BuyHints.jpg}	
		
			\subsubsection{The Quit Challenge}
		
			\begin{itemize}
	
		
		\item Pre-Conditions
			\begin{enumerate}
				
				\item User must be logged in as a player
				\item Player should have started the challenge
			\end{enumerate}
		\item Post-Conditions
			\begin{enumerate}
			\item Challenge is stopped. 
						
			\end{enumerate}
		

		\end{itemize}
		
		
		\subsubsection* {The quit challenge use case diagram}
		\begin{itemize}
			\item Description\\
			This use case will be used by the REST clients, specifically the mobile app(tablet app), to quit the cahllenge during gameplay on the system.
		\end{itemize}	
		
		
		\includegraphics[width=14cm,height=6cm,keepaspectratio]{quitChallenge.jpg}
		
				\subsubsection{The Logout}
		\begin{itemize}
	
		
		\item Pre-Conditions
			\begin{enumerate}
				\item The user is succesfully logged into the system 
			
			\end{enumerate}
		\item Post-Conditions
			\begin{enumerate}
			\item The user will be logged out of the system
						
			\end{enumerate}
	

\end{itemize}

		\subsubsection* {The logout use case diagram}
		\begin{itemize}
			\item Description\\
			This use case will be used by the REST clients, specifically mobile app(tablet app), to logout of the system.
		\end{itemize}
		
	
		
	
		
		\subsection{Activity Diagrams}		
		
		\subsubsection* {The Login activity diagram.}
		\includegraphics[width=10cm,height=6cm,keepaspectratio]{Model.jpg}		
		
		\subsubsection* {The LogOut activity diagram.}
		\includegraphics[width=10cm,height=6cm,keepaspectratio]{LogOut.jpg}	
		
			\subsubsection* {The View Challenges activity diagram.}
			\includegraphics[width=10cm,height=6cm,keepaspectratio]{ViewChallenges.jpg}	
		
		 \subsubsection* {The Add user activity diagram.}
		 \includegraphics[width=10cm,height=6cm,keepaspectratio]{addUser1.jpg}	
		 
		 \subsubsection* {The Remove user activity diagram.}
		 \includegraphics[width=10cm,height=6cm,keepaspectratio]{RemoveUser1.jpg}	
		 
		 \subsubsection* {The View Challenges activity diagram.}
		 \includegraphics[width=10cm,height=6cm,keepaspectratio]{ViewChallenges.jpg}
		 
		  \subsubsection* {The Start Challenge activity diagram.}
		  \includegraphics[width=10cm,height=6cm,keepaspectratio]{StartChallenge1.jpg}
		  
		   \subsubsection* {The Change keywords activity diagram.}
		   \includegraphics[width=10cm,height=6cm,keepaspectratio]{ChangeKeywords.jpg}
		 
		 
		 \subsection{Domain Model}
		 \includegraphics[width=15cm,height=15cm,keepaspectratio]{domainModel.png}
			
	\fi
		

	\newpage
	\section{Open Issues}		


		
\end{document}
