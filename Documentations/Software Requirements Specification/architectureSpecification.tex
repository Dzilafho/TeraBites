\documentclass[english]{article}


\usepackage{graphicx}
\usepackage{grffile}
\usepackage[T1]{fontenc}
\usepackage{babel}



\title{\scshape\Large Team members}
\author{
	NG Maluleke\\
	\texttt{13229908}
	\and
	D Mulugisi\\
	\texttt{13071603}
	\and
	C Nel\\
	\texttt{14029368}
	\and
	LE Tom\\
	\texttt{13325095}
}


\graphicspath{{Pictures/}}

\begin{document}

	
	\begin{figure}
	      \includegraphics[width=\linewidth]{up_logo.png}
	\end{figure}
	
	\begin{center}
	 \line(1,0){370}
	\\[0.2cm]
    {\scshape\Large Architecture specification \par}
	\vspace{0.1cm}
	\line(1,0){370}
	\\[0.8cm]
	
	{\scshape\large Project name: Arcane Arcade\par}	
	\vspace{1cm}
	{\scshape\large Client: Tony vd Linden\par}
	\vspace{1cm}
	{\scshape\large Team name: Terabites\par}
	\vspace{1cm}
	{\let\newpage\relax\maketitle}
	\end{center}
	
	
	\pagenumbering{gobble}
	\newpage
	\tableofcontents

	\pagenumbering{arabic}
	\newpage
	
	\section{Introduction}
		Our project \textit{Arcane Arcade} has a vision to provide a profoundly manner of testing the abilities of developers, who are potential employees, and using the results of the captured
		data from these tests to be able to assign them accordingly to a specific postion within the company which is best suited for their skills.\\\\
		Since the core focus is on software developers, \textit{Arcane Arcade} does testing on the programming abilities of the candidates using an esoteric language (esolang). Candidates are 
		provided with a series of programming challenges and questions which they have to complete using the esoteric language.


	\section{Architecture Requirements}
		\subsection{Access channel requirements}
			\paragraph\indent
			The system requires two interfaces:
			\begin{list}{$\bullet$}{\leftmargin=1.5cm \itemindent=0em}
				\item Web interface (Maintenance portion)
				\item It is preferred that the game portion be built to run on a tablet device
			\end{list}


		\subsection{Quality requirements}
			\paragraph\indent
							The following quality aspects needs to be addressed:
			\begin{list}{$\bullet$}{\leftmargin=1.5cm \itemindent=0em}
				\item \textbf{Perfomance:} Performance is the most important quality requirement for this application. This requirement pertains to how fast the application responds to certain actions within a set time interval. The application will use a compiler instead of an interpreter which will translate the esoteric language into object code, which performs better than an interpreter.
				
				\item \textbf{Reliability:} Reliability is the second most important quality requirement. Care should be given to make the application as reliable as possible without sacrificing performance. The application should have maximum uptime and proper fault tolerance. The application should help the user in avoiding errors, such as submitting incompatible values.
				
				\item \textbf{Maintainability:} Maintainability can be achieved by properly documenting the source code. Loose coupling will make it easy to completely change the esoteric language (by simply changing the ANTLR grammar file and the parser) and the front end of the program as long as it conforms to the specified API calls.
				
				\item \textbf{Scalability:} The system should be able to service at least one administrator and a few users (potential employees), and thus it is required for processes to be able to execute concurrently, which the GlassFish server enables.
				
				\item \textbf{Security:} Only authenticated API calls should be allowed and user details should be properly encrypted.			
				
				\item \textbf{Cost:} Since some users will be accessing the application via mobile internet, low bandwidth communication is essential. Thus, communication will be done via JSON instead of XML.
				
			\end{list}
			
			\subsection{Architecture styles}


			
		\subsection{Integration requirements}
			\par The system will be primarily used via a web interface which will interact with the server through a RESTful API, decoupling the client from the server as best as possible.
			
		\setcounter{secnumdepth}{5}
		\subsection{Architecture constraints}
			\subsubsection{Technologies}
			\begin{itemize}
		  \item HTML and JavaScript for the front end functionality for both the browser and game application
		  \item Java Standard Edition (J2SE) will be used for backend functionality.
		  \item PostgreSQL for persistence because it is a mature, efficient and reliable relational database
          implementation which is available across platforms and for which there is a large and very competent
         support community.
       \end{itemize}
       
		\subsubsection{Architectural patterns/frameworks} % level 1
		\textbf{Three-tier Architecture} % level 2
		The main reason for using three-tier architecture is to allow any of the three tiers to be upgraded or replaced independently, when changes in requirements or technology require such upgrades or replacements. Thus addressing \textbf{maintainability} of the application. By having a dedicated application-tier which can run on a capable server, \textbf{performance} is also addressed since the client side is freed from the more demanding computations such as the compilation of the user code.
		
		\vspace{0.5cm}		
		
		\textbf{The three tiers are:}
			\begin{itemize}
			\item \textbf{Presentation-tier:} This will be represented the front-end of the management or maintenance portal which will communicate with other tiers by sending results to the browser and other tiers in the network.
			\item \textbf{Application-tier:} This is the logical tier which provides the application's functionality.	
			\item \textbf{Data tier:} All the data persistence mechanisms which will be used. Data in this tier is kept independent of application servers or business logic. 	
			\end{itemize}	
			
			\vspace{0.5cm}
    \newpage
	\textbf{Frameworks} % level 2

	
	\textbf{AngularJS}
	\begin{itemize}
		\item 	Which is a powerful flexible framework with good modularization and extensive module
		repositories. With Angular.js there is, however, a higher risk for the application to degenerate
		into an inconsistent and difficult to maintain web application as it does not enforce a consistent
		way of developing the application. The two-way binding
		and the updating of athe virtual DOM is also less efficient in Angular.js.
		\item REST (Representational State Transfer)

	\end{itemize}
	
	
		
	
\end{document}
