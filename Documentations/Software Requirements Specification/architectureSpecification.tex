\documentclass[english]{article}


\usepackage{graphicx}
\usepackage{grffile}
\usepackage[T1]{fontenc}
\usepackage{babel}



\title{\scshape\Large Team members}
\author{
	NG Maluleke\\
	\texttt{13229908}
	\and
	D Mulugisi\\
	\texttt{13071603}
	\and
	C Nel\\
	\texttt{14029368}
	\and
	LE Tom\\
	\texttt{13325095}
}


\graphicspath{{Pictures/}}

\begin{document}

	
	\begin{figure}
	      \includegraphics[width=\linewidth]{up_logo.png}
	\end{figure}
	
	\begin{center}
	 \line(1,0){370}
	\\[0.2cm]
    {\scshape\Large Architecture specification \par}
	\vspace{0.1cm}
	\line(1,0){370}
	\\[0.8cm]
	
	{\scshape\large Project name: Arcane Arcade\par}	
	\vspace{1cm}
	{\scshape\large Client: Tony vd Linden\par}
	\vspace{1cm}
	{\scshape\large Team name: Terabites\par}
	\vspace{1cm}
	{\let\newpage\relax\maketitle}
	\end{center}
	
	
	\pagenumbering{gobble}
	\newpage
	\tableofcontents

	\pagenumbering{arabic}
	\newpage
	
	\section{Introduction}
	The name of our project is Arcane Arcade which references the esoteric language that users will have to use to interact with the system. It also highlights the gamification approach that makes it a fun activity. The platform will be used to test prospective employees in a fun gamified way. It will indicate in which programming area the user is likely to be better suited, by looking at how the user completed the challenges, such as whether the user used any hints and the badges that the user earned. Users will be presented with several challenges which they will have to complete and based on their performance,placed on appropriate positions within the company.
	
	\section{Vision}
	The vision is to provide a fun and easily accessible platform that can be used to gauge the programming aptitude and capabilities of existing or future software developers using a custom esoteric programming language

	\section{Architecture Requirements}
		\subsection{Access channel requirements}
			\paragraph\indent
			The system requires two interfaces:
			\begin{list}{$\bullet$}{\leftmargin=1.5cm \itemindent=0em}
				\item The user will have access to the system through Web interface (for both the game and maintenance portal) on any preferred browser.
				\item The system will also be made available to the user on a Tablet application.
			\end{list}


		\subsection{Quality requirements}
			\paragraph\indent
							The following quality aspects needs to be addressed:
			\begin{list}{$\bullet$}{\leftmargin=1.5cm \itemindent=0em}
				\item \textbf{Perfomance:} Performance is the most important quality requirement for this application. This requirement pertains to how fast the application responds to certain actions within a set time interval. The application will use a compiler instead of an interpreter which will translate the esoteric language into object code, which performs better than an interpreter.
				
				\item \textbf{Reliability:} Reliability is the second most important requirement and will be achieved by implementing user validation on inputs so that they can't submit unexpected values. Fault tolerance is also needed, such as rolling back changes when errors occur.
				
				\item \textbf{Maintainability:} Maintainability can be achieved by properly documenting the source code. Loose coupling will make it easy to completely change the esoteric language (by simply changing the ANTLR grammar file and the parser) and the front end of the program as long as it conforms to the specified API calls.
				
				\item \textbf{Scalability:} The system should be able to service at least one administrator and a few users (potential employees), and thus it is required for processes to be able to execute concurrently, which the GlassFish server enables.
				
				\item \textbf{Security:} Only authenticated API calls should be allowed and user details should be properly encrypted.			
				
				\item \textbf{Cost:} Since some users will be accessing the application via mobile internet, low bandwidth communication is essential. Thus, communication will be done via JSON instead of XML.
				
			\end{list}
			
			\subsection{Architecture styles}


			
		\subsection{Integration requirements}
			\par The system will be primarily used via a web interface which will interact with the server through a RESTful API, decoupling the client from the server as best as possible.
			
		\setcounter{secnumdepth}{5}
		\subsection{Architecture constraints}
			\subsubsection{Technologies}
			\begin{itemize}
		  \item HTML
		  \item JavaScript
		  \item CSS
		  \item JQUERY
		  \item Java Standard Edition (J2SE)
		  \item PostgreSQL for persistence because it is a mature, efficient and reliable relational database
          implementation which is available across platforms and for which there is a large and very competent
          support community.\\ 
         \newline
          PostgreSQL satistifies the Scalability requirement rather nicely. It can handle a variety of different workloads, from single machine applications to much larger web applications with many concurrent users. This means that the program can be easily used to ensure that our sysstem can scale to the workload that is required of it. 
       \end{itemize}
       
		\subsubsection{Architectural patterns} % level 1
		\begin{itemize}
			\item \textbf{Model-View-Controller} is the main architectural pattern used. The \textit{model} refers to the component on the server which manages the data and logic of the application, such as the web server and compiler. The \textit{view} refers to the output on the front-end of the application, ie. on the web and mobile interface. The \textit{controller} is the part which accepts the user input and changes it into the appropriate commands, or API calls between the model and the view components.
		\end{itemize}
		
		\iffalse
		\textbf{Three-tier Architecture} % level 2
		The main reason for using three-tier architecture is to allow any of the three tiers to be upgraded or replaced independently, when changes in requirements or technology require such upgrades or replacements. Thus addressing \textbf{maintainability} of the application. By having a dedicated application-tier which can run on a capable server, \textbf{performance} is also addressed since the client side is freed from the more demanding computations such as the compilation of the user code.
		
		\vspace{0.5cm}		
		
		\textbf{The three tiers are:}
			\begin{itemize}
			\item \textbf{Presentation-tier:} This will be represented the front-end of the management or maintenance portal which will communicate with other tiers by sending results to the browser and other tiers in the network.
			\item \textbf{Application-tier:} This is the logical tier which provides the application's functionality. This includes the compiler component of the application.	
			\item \textbf{Data tier:} All the data persistence mechanisms which will be used. Data in this tier is kept independent of application servers or business logic. 	
			\end{itemize}	
			
			\vspace{0.5cm}
			
		\fi
	\subsubsection{Architectural Frameworks} % level 2

	\begin{itemize}
		
		
	\item \textbf{ANTLR 4} \\
	ANTLR 4 is used to generate a lexer and parser for the esoteric language using a grammar file as input. A visitor implementation is also created to deal with the tokens created from an input string (the esolang code) and calculate the result of the user's code.
	
	\item \textbf{AngularJS} \\ 
	Angular js is used for dependency injection. It allows us to speficify fields into which data from the database can simply be placed without the need to explicitly know the data or.
	
	\item \textbf{Gson}\\ Gson will be used for the serialization and deserialization of Java objects JSON. It aids in creating JSON objects that encapsulate all the fields and relations that the Java object we are serializing has.
	\item \textbf{Java Jersey} \\
	Java Jersey will be used for mapping a resource classes (POJOs) as a web resources. It allows us to use annotations  when creating web services according to the Representational State Transfer (REST) architectural pattern.This framework is considerably useful in our situation as we are using both Java and a RESTful API approach to making the system.
		\end{itemize}	

	\iffalse
	\textbf{AngularJS}
	\begin{itemize}
		\item 	Which is a powerful flexible framework with good modularization and extensive module
		repositories. With Angular.js there is, however, a higher risk for the application to degenerate
		into an inconsistent and difficult to maintain web application as it does not enforce a consistent
		way of developing the application. The two-way binding
		and the updating of athe virtual DOM is also less efficient in Angular.js.
		\item REST (Representational State Transfer)

	\end{itemize}
	\fi
	
\end{document}
