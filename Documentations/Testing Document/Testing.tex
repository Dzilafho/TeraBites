\documentclass[english]{article}


\usepackage{graphicx}
\usepackage{grffile}
\usepackage[T1]{fontenc}
\usepackage{babel}



\title{\scshape\Large Team members}
\author{
	NG Maluleke\\
	\texttt{13229908}
	\and
	D Mulugisi\\
	\texttt{13071603}
	\and
	C Nel\\
	\texttt{14029368}
	\and
	LE Tom\\
	\texttt{13325095}
}


\graphicspath{{Pictures/}}

\begin{document}

	
	\begin{figure}
		\includegraphics[width=\linewidth]{up_logo.png}
	\end{figure}
	
	\begin{center}
	 \line(1,0){370}
	\\[0.2cm]
    {\scshape\Large Testing Document \par}
	\vspace{0.1cm}
	\line(1,0){370}
	\\[0.8cm]
	
	{\scshape\large Project name: Arcane Arcade\par}	
	\vspace{1cm}
	{\scshape\large Client: Tony vd Linden\par}
	\vspace{1cm}
	{\scshape\large Team name: Terabites\par}
	\vspace{1cm}
	{\let\newpage\relax\maketitle}
	\end{center}
	
	
	\pagenumbering{gobble}
	\newpage
	\tableofcontents

	\pagenumbering{arabic}
	\newpage
	
	\section{Introduction}
		 The project is called \textit{Arcane Arcade}, which references the esoteric language users will have to use, as well as the gamification approach to try and make it as fun as possible.

	\section{Objectives and Tasks}
		\subsection{Objectives}
		\subsection{Tasks}
		
	\section{Testing Strategy}
		\subsection{System Test}
		\subsection{Performance Test}
		\subsection{Security Test}
		\subsection{Automated Test}
		\subsection{Stress and Volume Test}
	
	\section{Execution Strategy}
		\subsection{Entry and Exit Criteria}
		\subsection{Test Cycles}
		\subsection{Dependencies}
			\subsubsection{Hardware}
			\subsubsection{Software}
			\subsubsection{Data}
			
	\section{Test Management Procedures}
		\subsection{Result Reviews}
		\subsection{Defect Reporting}
		\subsection{Change Reporting}
		
	\section{Test Environment}
		
\end{document}
