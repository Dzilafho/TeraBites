\documentclass[english]{article}


\usepackage{graphicx}
\usepackage{grffile}
\usepackage{float}
\usepackage[T1]{fontenc}
\usepackage{babel}

\setcounter{secnumdepth}{4}

\title{\scshape\Large Team members}
\author{
	NG Maluleke\\
	\texttt{13229908}
	\and
	D Mulugisi\\
	\texttt{13071603}
	\and
	C Nel\\
	\texttt{14029368}
	\and
	LE Tom\\
	\texttt{13325095}
}


\graphicspath{{Pictures/}}

\begin{document}

	
	\begin{figure}
		\includegraphics[width=\linewidth]{up_logo.png}
	\end{figure}
	
	\begin{center}
	 \line(1,0){370}
	\\[0.2cm]
    {\scshape\Large User Manual \par}
	\vspace{0.1cm}
	\line(1,0){370}
	\\[0.8cm]
	
	{\scshape\large Project name: Arcane Arcade\par}	
	\vspace{1cm}
	{\scshape\large Client: Tony vd Linden\par}
	\vspace{1cm}
	{\scshape\large Team name: Terabites\par}
	\vspace{1cm}
	{\let\newpage\relax\maketitle}
	\end{center}
	
	
	\pagenumbering{gobble}
	\newpage
	\tableofcontents

	\pagenumbering{arabic}
	\newpage
	
	\section{Introduction}
		 The project is called \textit{Arcane Arcade}, which references the esoteric language users will have to use, as well as the gamification aspect.
		 \\[12pt]
		It is used by employers to determine the level of skills and classification if potential employees. 
		 \\[12pt]
		 This document presents the Arcane Arcade system, it's uses and how it is to be put to use.

	\section{Project Background}
	This section outlines the background behind the project Arcane Arcade.
		\subsection{Problems Experienced}
		The software engineering industry have tried various models over the years to ascertain whether a prospective employee has the required skills or aptitude to be a valuable software developer. These range from automated online screening assessments that pushes the boundaries of the person's intellectual capability, to manually executed proprietary company tests using specific programming languages to test development competencies. Online assessments are popular and easy to execute, but lacks the insight provided by manual and proprietary testing of how the candidate reasons, where manual and proprietary testing is time-consuming and expensive to execute. In addition, company proprietary tests become stale and are leaked into the industry reducing the value tests may have.
		\\[12pt]
		BBD have used both approaches over the years and has found that online assessments are adequate when the prospective employee has little to no professional development experience. This is however not enough to demonstrate software development capability for existing professional developers and the company was forced to create a programming test for this group of people. These days, it is becoming difficult to maintain a landscape for an array of different programming languages used by BBD, while ensuring that the quality is maintained.

		\newpage
		\subsection{Arcane Arcade as the Solution}
		The candidate is presented with a problem and must use an unfamiliar and custom esoteric programming languages (esolang) to solve the problem. The test must make use of gamification principles such as making new keywords available that can be used in following challenges when a specific challenge is completed. 
		\\[12pt]
		Each challenge is harder than the previous and all incorporate some programming principle such as general arithmetic, condition checking, iteration, recursion, functions and embedded functions. The platform must have the ability to interpret or compile the esolang and to allow the candidate to run the program. The platform must be available on-line or on a tablet device and be secure enough to disallow unauthorized use. The candidate will be emailed a link and a user-name to access the platform. 
		\\[12pt]
		All the challenges are be timed and all results and times must be persisted. Once the candidate completes the whole challenge or decides to stop before the last challenge, a final score is shown. The candidate cannot access the challenge again once it reaches this state. In addition to the challenge/game, an administrator must be able to manage users, questions, esolang keywords (dependent on time) and result on-line.
		
		
	\section{System Overview}
	The project is called the Arcane Arcade and the name references both gaming as well as esoteric (intended for or likely to be understood by only a 		small number of people with a specialized knowledge or interest, arcane: mysterious or understood by few) programming languages.
	
	\section{System Configuration}
	The system requires a Windows/Unix based host to run the server. This
host must have the associated technologies installed (the installation of these
technologies will be discussed). The host must be connected to the
internet in order to allow any required dependencies to be installed and set
up for the operating system environment. The configuration of the server
requires an active internet connection to facilitate communication between the
system and end users.\\
\newline
 The challenges that are to be passed to the candidate's system reside in the database, there will have to be retrieved for display in the pages. The levels and questions that the challenges entail will be stored too. The challenges that the system houses can be added to the system by the administrator.The administrators are the only users who are able to  edit any data in the database.
\newpage
 The users will only be allowed to view and play the challenges on the game page. The user's sign up information will also be stored and will be used on the user's profile page. Each user will only have access to their own information. The administrators, however, will be able to search for users and update their IDs or delete them from the system. The administrators will have access to the challenge history of each user in order to view their attempts and performance.
 \newline

	\begin{figure}[H]
	  
	 	\includegraphics[width=1.0\textwidth]{systemOverview.png}
	 	\caption{Overview of System - Servers and Clients}
	\end{figure}
 \newpage
 
 \section {Installation}
 
 \subsection{Prerequisites}
 For the programmer, who will maintain the code: \\ 
 \newline
 The Arcane Arcade System must have the associated technologies, PosgreSQL,\\AngularJs,Maven and Glassfish web server.The programmer will also require an IDE most preferably Netbeans for further development and maintenance purposes. Arcane Arcade is a maven project meaning you will need an Internet connection to download all the required project dependencies from remote sites to your machine in order for your project to build succesfully\\
 
The following command can be used to clean the project from the terminal
 \begin{itemize}
	 
	 \item mvn clean
\end{itemize}

 The user can also clean the project with Netbeans as shown below\\
 
 	\begin{figure}[H]
 		
 		\includegraphics[width=1.0\textwidth]{clean.png}
 		\caption{Cleaning the project machine files}
 	\end{figure}
  
 
After cleaning the project you have to build. This will download all the dependencies declared on the project POM file from a remote locations into your local machine.
Some of these dependencies include Java Jersey fro REST services etc. \\

The following command can be used to build the project from the terminal
  \begin{itemize}
  	
  	\item mvn build
  	
  \end{itemize}
  The user can also build the project with Netbeans as shown below\\
  
  \begin{figure}[H]
  	
  	\includegraphics[width=1.0\textwidth]{build.png}
  	\caption{Building the project machine files}
  \end{figure}

  
  Once the project has been built one has to deploy it on glassfish server which comes with the Netbeans installation or can be downloaded separately on the glassfish website.
  The project can also be deployed on any other web server depending on the user's preference.
  \begin{figure}[H]
    \includegraphics[width=1.0\textwidth]{glassfish.png}
  	\caption{Server used to deploy the project}
  \end{figure}
    
  
	Once the server is set up we deploy and run the project on Netbeans using glassfish server and the project is deployed on localhost or on the local machine and can be accessed with the command below on the browser.
  
    \begin{itemize}
    	
    	\item http://localhost:10530/ArcaneArcade/
    	
    \end{itemize}
  
  The screen shot below shows below shows the project while being deployed on glassfish server on Netbeans.
   \begin{figure}[H]
   	\includegraphics[width=1.0\textwidth]{deploy.png}
   	\caption{Server used to deploy the project}
   \end{figure}
 
 
	
	\section{System Usage}
	
	Access to Arcane Arcade is through a standard web browser. Different types of users have access to different facets of the system. The system has an administrator user account, who is responsible for the management of the entire system and they also have global access to the whole system. The system also has a player account which can only access the game part of the system to complete the challenges
	\newline

	Any user that wishes to use Arcane Arcade will need an Internet connection and a web browser to access the website. Any operating system will run fine and the website is accessable from The following browsers: 
	\begin{itemize}
		\item Google
		\item Chrome Mozilla 
		\item FireFox 
		\item Internet Explorer 
		\item Safari 
	\end{itemize}
	If your browser is not listed here, the site may work with limited functionality.\\
	
Administrators have global access to the whole system which also grants them management capabilities. The administrators can modify the structure of the challenges and how the candidates interact with the system controlling access, modifying questions, adding challenges, and others.
		\\[12pt]
	Candidates only have access to the challenges page and all-sub pages that can be found/clicked from that particular page.
	
	
		\subsection{Administrative Use}
		Arcane Arcade initially starts with a couple of administrator user accounts. These accounts have access to the management web page which can be used to cater for the following.\newline
			\subsubsection{User Management}
				\includegraphics[width=\linewidth]{User.png}				\newline

				The very first section in the administration page is user management. This is used to manage the user accounts related to the system.
				\\[10pt]\newline
				
				\newpage
				\paragraph{Add User}. \\ \newline
				\includegraphics[width=\linewidth]{AddUser.png}				\newline

Once the form is completely and correctly filled in and the type of user has been selected clicking on the Add User button creates the user account and the system will display an appropriate response.
				\\[10pt]
				
				\paragraph{View Users}. \\ \newline
				\includegraphics[width=\linewidth]{ViewUsers.png}				\newline
					To view all the accounts which have access to the system, one has to click on the View Users.\newline
				
				\paragraph{Remove User}. \\ \newline
				\includegraphics[width=\linewidth]{RemoveUser.png}				\newline

					To remove an existing user, the administrator simply enters the person's special unique user-name in the field under a Remove User heading.If the specified user-name is associated with an account then the account is removed from the system and access is revoked.
					\\[12pt]\newline
				
				
				
				
					
			
			
			
			\newpage	
			\subsubsection{Challenge Management}
				\includegraphics[width=\linewidth]{Challenge.png}				\newline
					Challenges are the hosts of the smaller more interactive elements of the system. Challenges house the Levels which in turn house the Questions.				\\[12pt]\newline

				\paragraph{Add Challenge}. \\ \newline
				\includegraphics[width=\linewidth]{AddChallenge.png}				\newline
To add a new Challenge you simply specify the Challenge name and click on the Add Challenge button. This creates an  empty challenge into which you can add levels.
				\\[12pt]\newline

				\paragraph{View Challenges}. \\ \newline
				\includegraphics[width=\linewidth]{ViewChallenges.png}				\newline
				To view a list of all available challenges you would click on the View Challenges button.\newline


				\paragraph{Remove Challenge}. \\ \newline
				\includegraphics[width=\linewidth]{RemoveChallenge.png}				\newline
				Removing a non-empty Challenge removes also all it's levels and those levels' questions. You enter the name of the challenge and click on remove challenge.
				\\[12pt]
			
				
				
				
			\subsubsection{Level Management}
				\includegraphics[width=\linewidth]{Level.png}				\newline
				The very first section in the administration page is user management. This is used to manage the user accounts related to the system.
				\\[12pt]\newline
				
				\paragraph{Add Level}. \\ \newline
				\includegraphics[width=\linewidth]{AddLevel.png}				\newline
To add a new Level you fill in the form under Level Management by entering the name of the new Level and the Challenge that it falls under. Clicking the Add Level button adds the Level under the specified Challenge.
				\\[12pt]\newline

				\paragraph{View Levels}. \\ \newline
				\includegraphics[width=\linewidth]{ViewLevels.png}				\newline
				To view all the different Levels in table format you simply click the View all Levels button.\newline

				\paragraph{Remove Level}. \\ \newline
				\includegraphics[width=\linewidth]{RemoveLevel.png}				\newline
To remove a Level you simply specify the name of the level then click on Remove Level. The level will then be removed from the Challenge which it was under.
				\\[12pt]
				
				
			
			
			\subsubsection{Question Management}
				\includegraphics[width=\linewidth]{Question.png}				\newline

					\paragraph{Add Question}. \\ \newline			
				\includegraphics[width=\linewidth]{AddQuestion.png}				\newline
Adding a new question requires completion of the form under the Add Questions heading. The form has to be filled in completely and correctly, this is achieving by providing the Question, the Answer to the Question, the Level under which the question falls and the Challenge. Clicking on the Add Question button then puts it in the system accordingly.
				\\[12pt]\newline

\paragraph{View Questions}. \\ \newline
				\includegraphics[width=\linewidth]{ViewQuestions.png}				\newline


\paragraph{Remove Question}. \\ \newline
				\includegraphics[width=\linewidth]{RemoveQuestion.png}				\newline
				To view the list of all Questions in the system you click on the View Questions button.\newline


				
			
			\newpage
		\subsection{Candidate Use}
			\subsubsection{Login}
				\includegraphics[width=\linewidth]{Login.jpg}
				Upon opening the address to the Arcane Arcade system the user is greeted with a login screen. The login screen asks for the user's user-name and password. 
				\\[12pt]
				You fill in the two fields with required registered information then click on the Login button to get access to the system.
			
\newpage			
			\subsubsection{Answer Question}
				\includegraphics[width=\linewidth]{question.jpg}
				When the user successfully logs in they are directed to a menu which presents them with a question. The user has to then answer that question as it is used by typing in a series of character that is recognized as the Arcane Arcade esoteric language.
			\subsubsection{View Code Output}
				\includegraphics[width=\linewidth]{outResults.jpg}
				After inserting the question string it gets interpreted and executed. The result page then shows what the esoteric language question means in regular or well known syntax. The user has click on the send button in the middle of the screen.
		
	\section{Troubleshooting}
		\subsection{Administrative Problems}
		\subsection{Candidate Problems}
		
\end{document}
