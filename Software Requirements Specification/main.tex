\documentclass[english]{article}


\usepackage{graphicx}
\usepackage{grffile}
\usepackage[T1]{fontenc}
\usepackage{babel}



\title{\scshape\Large Team members}
\author{
	NG Maluleke\\
	\texttt{13229908}
	\and
	D Mulugisi\\
	\texttt{13071603}
	\and
	C Nel\\
	\texttt{14029368}
	\and
	LE Tom\\
	\texttt{13325095}
}


\graphicspath{{Pictures/}}

\begin{document}

	
	\begin{figure}
		\includegraphics[width=\linewidth]{up_logo.png}
	\end{figure}
	
	\begin{center}
	 \line(1,0){370}
	\\[0.2cm]
    {\scshape\Large Software Requirements Specification \par}
	\vspace{0.1cm}
	\line(1,0){370}
	\\[0.8cm]
	
	{\scshape\large Project name: Arcane Arcade\par}	
	\vspace{1cm}
	{\scshape\large Client: Tony vd Linden\par}
	\vspace{1cm}
	{\scshape\large Team name: Terabites\par}
	\vspace{1cm}
	{\let\newpage\relax\maketitle}
	\end{center}
	
	
	\pagenumbering{gobble}
	\newpage
	\tableofcontents

	\pagenumbering{arabic}
	\newpage
	
	\section{Introduction}
		 The project is called \textit{Arcane Arcade}, which references the esoteric language users will have to use, as well as the gamification approach to try and make it as fun as possible.

	\section{Vision}
		The vision is to provide a fun and easily accessible platform that can be used to gauge the programming aptitude and capabilities of existing or future software developers using a custom esoteric programming language.

	\section{Background}
		There are many models in which to ascertain whether a prospective employee has the required skills or aptitude to be a valuable software developer. Online assessments are easy to execute, but lack the insight provided by manual and proprietary testing of how the candidate reasons. Manual and proprietary testing on the other hand is time-consuming and expensive to execute. In addition, company proprietary tests become stale and are leakined into the industry reducing the value tests may have.
		
		BBD has thus opted to use an esoteric programming language in order to test the skills and aptitude of prospective employees, regardless of their programming experience or preferred programming language. A mobile and online platform is thus required to test prospective employees while keeping the tests fun by using gamification principles. The tests should also indicate in which area the user is likely to be better suited, by looking at how the user completed the challenges, such as whether the user used any hints.

	\newpage
	\section{Architecture Requirements}
		\subsection{Access channel requirements}
			\paragraph\indent
			The system requires two interfaces:
			\begin{list}{$\bullet$}{\leftmargin=1.5cm \itemindent=0em}
				\item Web interface
				\item Mobile app (Could also be a mobile friendly website)
			\end{list}


		\subsection{Quality requirements}
			\paragraph\indent
							The following quality aspects needs to be addressed:
			\begin{list}{$\bullet$}{\leftmargin=1.5cm \itemindent=0em}
				\item \textbf{Perfomance:} Performance is the most important quality requirement for this application. This requirement pertains to how fast the application responds to certain actions within a set time interval. The application will use a compiler instead of an interpreter which will translate the esoteric language into object code, which performs better than an interpreter.
				\item \textbf{Scalability:} The system should be able to service at least one administrator and a few users (potential employees).
				\item \textbf{Security:} The system should be secure enough so that no unauthorized users will be able to access it and data integrity should be maintained.
				\item \textbf{Reliability:} The system should be reliable in all functions, especially when compiling the user's code in order to avoid wrong results.
				\item \textbf{Maintainability:} Maintainability is amongst the most important quality requirements for the application. Since the future of the application may require total change of the esolang, developers should be able to easily and relatively quickly change aspects of the functionality the system provides or add new functionality to the system. This means that care should be given to modularity.
				\item \textbf{Cost:} Since some users will be accessing the application via mobile internet, care should be given to keep the required bandwidth to a minimum.
				
			\end{list}

			
		\subsection{Integration requirements}
			\par The system will be primarily used via a web interface which will use REST resources to interact with the layer providing the actual services.
			
		\setcounter{secnumdepth}{5}
		\subsection{Architecture constraints}
			\subsubsection{Technologies}
			\begin{itemize}
		  \item HTML and JavaScript for the front end functionality for both the browser and game application
		  \item Java will be used for backend functionality.
		  \item PostgreSQL for persistence because it is a mature, efficient and reliable relational database
          implementation which is available across platforms and for which there is a large and very competent
         support community.
       \end{itemize}
		\subsubsection{Architectural patterns/frameworks} % level 1
		\paragraph{Three-tier Architecture} % level 2
		The main reason for using three-tier architecture is to allow any of the three tiers to be upgraded or    			replaced independently, when changes in requirements or technology require such upgrades or           				replacements.Thus addressing \textbf{maintainability} the application.
		
		\vspace{0.5cm}
		
		
		\begin{itemize}
			\item \textbf{The three tiers in a three-tier architecture are:}
			\begin{itemize}
		\item \textbf{Presentation-tier:} This will be represented the front-end of the management or maintenance 			portal which will communicate with other tiers by sending results to the browser and other tiers in the       		network.
		\item \textbf{Application-tier:} This is the logical tier which provides the application's functionality.	
		\item \textbf{Data tier:} All the data persistence mechanisms which will be used.Data in this tier is kept independent of application servers or business logic. 	
	  \end{itemize}
				\end{itemize}
		\paragraph{Authentication Enforcer pattern} % level 3
		
		This pattern will help us improve security of the system while keeping
		the scalability and performance well maintained since the Authentica
		tion Enforcer pattern provides a consistent and structured way to handle
		authentication and verication of requests across actions within Web-tier
		components and also supports MVC architecture without duplicating the
		code.	
		
		
	
	\newpage
	\section{Functional requirements and application design}
		\subsection{Use case prioritization}
			\paragraph\indent
		
		\subsection{Use case/Services contracts}
			\paragraph\indent


		\subsection{Required functionality}
			\paragraph\indent


		\subsection{Process specifications}
			\paragraph\indent

		\subsection{Domain Model}
			\paragraph\indent

	\newpage
	\section{Open Issues}		
		\paragraph\indent

		
\end{document}
